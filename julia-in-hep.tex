% This paper is Copyright © Graeme Andrew Stewart and other
% authors, 2025.
% Licensed under Creative Commons Attribution 4.0 International (CC BY 4.0), see LICENSE

\documentclass{webofc}
\usepackage{graphicx} % Required for inserting images
\graphicspath{{figures/}}
\usepackage{subcaption}
\usepackage[varg]{txfonts}
\usepackage[utf8]{inputenc}

% Workaround for arXiv, start with "finalizecache" then switch to "frozencache"
\usepackage[finalizecache,cachedir=.]{minted}
% \usepackage[frozencache,cachedir=.]{minted}

\usepackage{natbib}
\usepackage{hyperref}
\usepackage{enumitem}

% For draft version
\usepackage{lineno}
\linenumbers

\newcommand{\unroot}{\texttt{UnROOT.jl}}

\newcommand{\kt}{${k}_\text{T}$}
\newcommand{\akt}{anti-${k}_\text{T}$}
\newcommand{\Akt}{Anti-${k}_\text{T}$}
\newcommand{\JR}{\texttt{JetReconstruction.jl}}
\newcommand{\ee}{$e^+e^-$}

\title{Julia in HEP}

\author{\firstname{Graeme Andrew} \lastname{Stewart}\inst{1}\fnsep\thanks{\email{graeme.andrew.stewart@cern.ch}} \and
\firstname{Sam} \lastname{Skipsey}\inst{2}
% etc.
}

\institute{CERN, Esplanade des Particules 1, Geneva, Switzerland
\and
\institute{School of Physics & Astronomy, University of Glasgow, Glasgow, United Kingdom, G12 8QQ}
}

\abstract{%
Julia is a mature general-purpose programming language, with a large ecosystem
of libraries and more than 10000 third-party packages, which specifically
targets scientific computing. As a language, Julia is as dynamic, interactive,
and accessible as Python with NumPy, but achieves run-time performance on par
with C/C++. In this paper, we describe the state of adoption of Julia in HEP,
where momentum has been gathering over a number of years.

HEP-oriented Julia packages can, via \texttt{UnROOT.jl}, already read HEP's
major file formats, including TTree and RNTuple formats. Interfaces to some of
HEP's major software packages, such as through \texttt{Geant4.jl}, are available
too. Jet reconstruction algorithms in Julia show excellent performance. A number
of full HEP analyses have been performed in Julia.

We show how, as the support for HEP has matured, developments have benefited
from Julia's core design choices, which makes reuse from and integration with
other packages easy. In particular, libraries developed outside HEP for
plotting, statistics, fitting, and scientific machine learning are extremely
useful.

We believe that the powerful combination of flexibility and speed, the wide
selection of scientific programming tools, and support for all modern
programming paradigms and tools, make Julia the ideal choice for a future
language in HEP.}

\begin{document}

\maketitle

\section{Programming Languages in High Energy Physics}
\label{sec:introduction}

\subsection{HEP Needs}

What do we need from code? Some desiderata.

\subsection{From Fortran to the C++/Python Era}

A bit of history from~\cite{pivarski2022}.

The situation today. Some thoughts on tradeoffs.

\section{Julia}

\subsection{Julia's Motivations}

Some of the design goals of the language. \emph{Julia Programming
Language}~\cite{bib:julia_freshapproach,10.1145/3276490}

\subsection{Julia in Practice}

Some code examples - demonstrate \emph{ease} and also give some benchmarks for
speed.

Tooling and ecosystem.

\subsection{Key Design Features for Performance}

Type system.

Multiple dispatch.

\section{Julia for Scientific Computing}

General adoption:~\cite{perkel-julia-science}.

GPU programming.

Some HPC codes.

\section{Julia in HEP}

\subsection{Challenges}

What does HEP need from its computing?

Cite some general overviews:~\cite{Stanitzki:2020bnx,eschle2023potential}.

\subsection{HEP Data Formats}

We can read that data... UpROOT.jl. EDM4hep.jl.

\subsection{Event Generators}

A bit about QuantumElectrodynamics.jl.

\subsection{Simulation}

Overview of Geant4.jl.

\subsection{Reconstruction}

JetReconstruction.jl

\subsection{Analysis}

Overview of analysis papers and suitability of Julia.

\subsection{End-to-end Computing}

The Legend Julia stack.

\section{Conclusions}

It's all good, nothing can go wrong. To infinity and beyond, etc.

\sloppy
\raggedright
% \clearpage
\bibliography{julia-in-hep}

\end{document}
